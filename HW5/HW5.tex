\documentclass{article}

% Language setting
% Replace `english' with e.g. `spanish' to change the document language
\usepackage[english]{babel}

% Set page size and margins
% Replace `letterpaper' with `a4paper' for UK/EU standard size
\usepackage[a4paper,top=2cm,bottom=2cm,left=3cm,right=3cm,marginparwidth=1.75cm]{geometry}

% Useful packages
\usepackage{amsmath}
\usepackage{graphicx}
\usepackage[colorlinks=true, allcolors=blue]{hyperref}
\usepackage{xcolor}
\usepackage{listings}

\colorlet{mygray}{black!30}
\colorlet{mygreen}{green!60!blue}
\colorlet{mymauve}{red!60!blue}

\lstset{
  backgroundcolor=\color{gray!10},  
  basicstyle=\ttfamily,
  columns=fullflexible,
  breakatwhitespace=false,      
  breaklines=true,                
  captionpos=b,                    
  commentstyle=\color{mygreen}, 
  extendedchars=true,              
  frame=single,                   
  keepspaces=true,             
  keywordstyle=\color{blue},      
  language=c++,                 
  numbers=none,                
  numbersep=5pt,                   
  numberstyle=\tiny\color{blue}, 
  rulecolor=\color{mygray},        
  showspaces=false,               
  showtabs=false,                 
  stepnumber=5,                  
  stringstyle=\color{mymauve},    
  tabsize=3,                                     
  title=\lstname 
}


\lstnewenvironment{code}[2][]{%
  \lstset{%
    numbers = left,
    title   = #2,
    #1,
  }%
}{}

\title{Homework 5}
\author{Steinarr Hrafn Höskuldsson}

\newcommand{\mycomment}[1]{}
\usepackage{fancyhdr}
\fancypagestyle{firststyle}
{
   \fancyhf{}
   \fancyhead[L]{Mechatronics 2}
   
   \renewcommand{\headrulewidth}{0pt} % removes horizontal header line
}
\begin{document}

\mycomment{

\begin{figure}[h]
    \centering
    \includegraphics[width=0.75\textwidth]{LAB3/Basic1.png}
    \caption{"Switch test" Breadboard set up}
    \label{fig:Switch_test}
\end{figure}

\lstinputlisting[caption=Defining 'ColorMatch' state, label={lst:colormatch}, language=Python, firstline=44, lastline=52]{LAB3/Basic.py}

} % end of comment

\pagestyle{firststyle}
{\let\newpage\relax\maketitle}

\section{Rudimentary Serial Communication}
I wrote a serial module myself from scratch when taking Embedded Systems last semester. I was able to reuse most of that code, however I moved some of it into macros just for fun. The code can be seen in Appendix \ref{appendix:code}

A program was written to test the serial library, it echos what is recieved excepts it adds one to the value. That is in order to  

\lstinputlisting[caption="main.c for testing the serial library"]{}

\section*{Final Project}

This week I researched options for sensors.
\begin{itemize}
    \item  For hand sensors (sensing when the hand strikes the thigh) my main 2 options are Piezo Contact Microphones and Accelerometers.

\item For the right foot sensor I am considering using an accelerometer mounted on the toe of the shoe, I theorize that it should experience a spike in acceleration when the foot strikes the ground.

\item For the left foot sensor I am also considering using an accelerometer. The left foot needs to sense the actual position of the foot since that constrols the sound from the HiHats. 

\end{itemize}
Next week I intend to source some simple accelerometers with analog outputs and some piexo contact microphones and test them out, try to measure the signal to noise ratio.

\newpage
\appendix
\section{Code}\label{appendix:code}

\lstinputlisting[firstline=25, caption=Arduino IDE version of blink program]{HW1/Blink/Blink.ino.ino}



\end{document}

